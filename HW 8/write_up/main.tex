\documentclass[12pt]{article}
\usepackage[utf8]{inputenc}

\usepackage[margin = 0.6in]{geometry}
\usepackage{amsmath}
\usepackage{amsfonts}
\usepackage{amssymb}
\usepackage{amsthm}
\usepackage{graphicx}
\usepackage{placeins}
\usepackage{enumitem}
\usepackage{dsfont}
\usepackage{booktabs}
\usepackage{subcaption}
\usepackage{pdflscape}

\newcommand{\N}{\mathbb{N}}
\newcommand{\Z}{\mathbb{Z}}
\newcommand{\R}{\mathbb{R}}
\newcommand{\E}{\mathbb{E}}
\newcommand{\Q}{\mathbb{Q}}
\newcommand{\de}{\mathrm{d}}
\newcommand{\one}{\mathds{1}}

\title{ECON 899b: Problem Set 1}
\author{Katherine Kwok\footnote{I collaborated with Anya Tarascina and Claire Kim on this assignment.}}
\date{November 2021}

\begin{document}

\maketitle
\noindent \textbf{Overview:} For this assignment, the goal is to apply different methods of optimization for discrete choice models. In particular, we use the public use microdata on mortgages to study the log-likelihood of the loan being pre-paid within the first-year. \\\\

\noindent \textbf{Findings:} The attached code (``main\_program.jl" and ``helper\_functions.jl") produces the results described below. In the first step, I wrote functions to evaluate log-likelihood, the score of the log-likelihood function (approximation of the FOC), and the Hessian matrix (approximation of SOC) given some $\beta$ vector. Specifically, $\beta_0 = -1$ and all other $\beta$ values in the vector are 0. Then, I use the numerical derivative method in Julia to calculate the FOC and SOC of the log-likelihood function at the same $\beta$ values. The comparison between the score and Hessian, and the numerical derivatives, are summarized in Tables \ref{tab2}, \ref{tab:first}, \ref{tab:second}. It appears that the approximations and numerical derivatives are very similar and/or identical. \\\\
Then, I wrote a Newton algorithm that solves for maximum likelihood, and compared the resulting coefficients to that of the BFGS (Quasi-Newton) and Simplex algorithms. Using the initial guess, the Newton algorithm converged in 37 iterations and approximately 12 seconds. The results of the algorithm are summarized in Table \ref{tab1} below. \\\\
The implementation of the BFGS algorithm was slightly more complicated. Without providing a gradient function, the algorithm took a very long time to converge. When I input the score of the log-likelihood as the gradient and used the resulting $\beta$ values from the Newton algorithm as the initial guess, the algorithm converged in 3 iterations and approximately 1 second. I used an educated guess for the initial value, because the simple initial values result in nonsensical results. \\\\
Similar to the BFGS algorithm, I was only able to get the Simplex algorithm to converge when I used the Newton algorithm results as the initial values. When I attempted to run the optimization without a refined initial guess, the optimization package returned a ``failed line search" message. With an educated guess, the Simplex algorithm converged in 694 iterations, and approximately 140 seconds. \\\
Comparing the $\beta$ results from all 3 algorithms, the coefficients appear to be very similar.

\begin{table}[!t]
    \centering \small
    \caption{Comparison of Log Likelihood Score and Numerical First Derivative}
    \begin{tabular}{cc}
    \par \toprule
    Score of Log Likelihood & Numerical First Derivative\\
    \par \midrule
	$-2605.91$ & $-2605.91$\\
	$-556.32$ & $-556.32$\\
	$-1156.86$ & $-1156.86$\\
	$-222.82$ & $-222.82$\\
	$-933.04$ & $-933.04$\\
	$-1215.13$ & $-1215.13$\\
	$-2109.63$ & $-2109.63$\\
	$-948.07$ & $-948.07$\\
	$-5049.88$ & $-5049.88$\\
	$-4534.79$ & $-4534.79$\\
	$-19401.9$ & $-19401.9$\\
	$-19164.66$ & $-19164.66$\\
	$-918.86$ & $-918.86$\\
	$-351.75$ & $-351.75$\\
	$-466.69$ & $-466.69$\\
	$-582.47$ & $-582.47$\\
	$-546.41$ & $-546.41$\\
    \par \toprule
    \end{tabular}
    \label{tab2}
\end{table}
\begin{table}[!b]
    \centering \small
    \caption{Comparison of Coefficients}
    \begin{tabular}{ccc}
    \par \toprule 
    $\beta_{newton}$ & $\beta_{bfgs}$ & $\beta_{simplex}$\\
    \par \midrule
    $-6.06$ & $-6.06$ & $-6.06$\\
    $0.87$ & $0.87$ & $0.87$\\
    $0.53$ & $0.53$ & $0.53$\\
    $0.6$ & $0.6$ & $0.6$\\
    $0.16$ & $0.16$ & $0.16$\\
    $0.87$ & $0.87$ & $0.87$\\
    $-0.06$ & $-0.06$ & $-0.06$\\
    $0.22$ & $0.22$ & $0.22$\\
    $1.01$ & $1.01$ & $1.01$\\
    $0.34$ & $0.34$ & $0.34$\\
    $-0.28$ & $-0.28$ & $-0.28$\\
    $0.19$ & $0.19$ & $0.19$\\
    $0.76$ & $0.76$ & $0.76$\\
    $1.15$ & $1.15$ & $1.15$\\
    $0.77$ & $0.77$ & $0.77$\\
    $0.38$ & $0.38$ & $0.38$\\
    $0.24$ & $0.24$ & $0.24$\\
    \par \toprule
    \end{tabular}
\label{tab1}
\end{table}



\begin{landscape}
    \begin{table}[!htbp]
    \centering \tiny
    \caption{Hessian Matrix for Log Likelihood Function}\label{tab:first}
    \begin{tabular}{ccccccccccccccccc}
\par \toprule
$-3224.6$ & $-880.4$ & $-1428.4$ & $-387.6$ & $-1305.7$ & $-1546.8$ & $-2619.4$ & $-1210.7$ & $-6304.6$ & $-5761.3$ & $-23783.2$ & $-23599.2$ & $-1405.0$ & $-664.4$ & $-681.9$ & $-674.4$ & $-583.0$\\
$-880.4$ & $-880.4$ & $-0.0$ & $-10.1$ & $-404.2$ & $-421.3$ & $-686.3$ & $-332.0$ & $-1720.7$ & $-1655.1$ & $-6608.0$ & $-6563.0$ & $-390.1$ & $-163.8$ & $-189.3$ & $-211.6$ & $-170.1$\\
$-1428.4$ & $-0.0$ & $-1428.4$ & $-165.2$ & $-560.3$ & $-676.0$ & $-1192.1$ & $-544.7$ & $-2796.0$ & $-2551.2$ & $-10512.7$ & $-10428.5$ & $-586.9$ & $-283.7$ & $-308.9$ & $-299.4$ & $-266.6$\\
$-387.6$ & $-10.1$ & $-165.2$ & $-715.6$ & $-185.7$ & $-187.9$ & $-325.0$ & $-152.3$ & $-783.9$ & $-694.0$ & $-2739.2$ & $-2721.0$ & $43.9$ & $-59.8$ & $-104.7$ & $-92.4$ & $-77.3$\\
$-1305.7$ & $-404.2$ & $-560.3$ & $-185.7$ & $-1305.7$ & $-693.5$ & $-973.1$ & $-501.4$ & $-2556.3$ & $-2592.7$ & $-9553.5$ & $-9586.9$ & $-502.0$ & $-214.5$ & $-291.0$ & $-299.4$ & $-192.9$\\
$-1546.8$ & $-421.3$ & $-676.0$ & $-187.9$ & $-693.5$ & $-806.4$ & $-1231.5$ & $-585.3$ & $-3024.7$ & $-2841.4$ & $-11435.2$ & $-11359.5$ & $-660.5$ & $-312.1$ & $-326.3$ & $-325.5$ & $-283.0$\\
$-2619.4$ & $-686.3$ & $-1192.1$ & $-325.0$ & $-973.1$ & $-1231.5$ & $-2224.7$ & $-992.5$ & $-5125.9$ & $-4620.8$ & $-19218.6$ & $-19050.9$ & $-1228.3$ & $-545.1$ & $-551.4$ & $-540.1$ & $-477.4$\\
$-1210.7$ & $-332.0$ & $-544.7$ & $-152.3$ & $-501.4$ & $-585.3$ & $-992.5$ & $-528.6$ & $-2369.9$ & $-2169.3$ & $-8869.9$ & $-8798.6$ & $-557.5$ & $-248.1$ & $-257.3$ & $-253.3$ & $-220.4$\\
$-6304.6$ & $-1720.7$ & $-2796.0$ & $-783.9$ & $-2556.3$ & $-3024.7$ & $-5125.9$ & $-2369.9$ & $-12464.4$ & $-11264.9$ & $-46482.7$ & $-46118.4$ & $-2718.9$ & $-1297.8$ & $-1333.2$ & $-1318.9$ & $-1134.6$\\
$-5761.3$ & $-1655.1$ & $-2551.2$ & $-694.0$ & $-2592.7$ & $-2841.4$ & $-4620.8$ & $-2169.3$ & $-11264.9$ & $-10834.7$ & $-42612.4$ & $-42303.3$ & $-2419.5$ & $-1169.8$ & $-1223.7$ & $-1213.9$ & $-1033.6$\\
$-23783.2$ & $-6608.0$ & $-10512.7$ & $-2739.2$ & $-9553.5$ & $-11435.2$ & $-19218.6$ & $-8869.9$ & $-46482.7$ & $-42612.4$ & $-176722.1$ & $-175070.8$ & $-10062.1$ & $-4908.2$ & $-5017.5$ & $-4970.1$ & $-4271.6$\\
$-23599.2$ & $-6563.0$ & $-10428.5$ & $-2721.0$ & $-9586.9$ & $-11359.5$ & $-19050.9$ & $-8798.6$ & $-46118.4$ & $-42303.3$ & $-175070.8$ & $-174101.2$ & $-9978.7$ & $-4851.0$ & $-4978.9$ & $-4944.8$ & $-4247.1$\\
$-1405.0$ & $-390.1$ & $-586.9$ & $43.9$ & $-502.0$ & $-660.5$ & $-1228.3$ & $-557.5$ & $-2718.9$ & $-2419.5$ & $-10062.1$ & $-9978.7$ & $-1405.0$ & $-293.0$ & $-306.9$ & $-305.7$ & $-268.4$\\
$-664.4$ & $-163.8$ & $-283.7$ & $-59.8$ & $-214.5$ & $-312.1$ & $-545.1$ & $-248.1$ & $-1297.8$ & $-1169.8$ & $-4908.2$ & $-4851.0$ & $-293.0$ & $-664.4$ & $-0.0$ & $-0.0$ & $-0.0$\\
$-681.9$ & $-189.3$ & $-308.9$ & $-104.7$ & $-291.0$ & $-326.3$ & $-551.4$ & $-257.3$ & $-1333.2$ & $-1223.7$ & $-5017.5$ & $-4978.9$ & $-306.9$ & $-0.0$ & $-681.9$ & $-0.0$ & $-0.0$\\
$-674.4$ & $-211.6$ & $-299.4$ & $-92.4$ & $-299.4$ & $-325.5$ & $-540.1$ & $-253.3$ & $-1318.9$ & $-1213.9$ & $-4970.1$ & $-4944.8$ & $-305.7$ & $-0.0$ & $-0.0$ & $-674.4$ & $-0.0$\\
$-583.0$ & $-170.1$ & $-266.6$ & $-77.3$ & $-192.9$ & $-283.0$ & $-477.4$ & $-220.4$ & $-1134.6$ & $-1033.6$ & $-4271.6$ & $-4247.1$ & $-268.4$ & $-0.0$ & $-0.0$ & $-0.0$ & $-583.0$\\
\par \toprule
\end{tabular}

    \bigskip
    \caption{Numerical Second Derivative}\label{tab:second} \par \medskip
    \begin{tabular}{cccccccccccccccc}
\par \toprule
$-880.3$ & $0.0$ & $-10.1$ & $-404.2$ & $-421.3$ & $-686.3$ & $-332.0$ & $-1720.7$ & $-1655.1$ & $-6608.0$ & $-6563.0$ & $-390.1$ & $-163.8$ & $-189.3$ & $-211.5$ & $-170.0$\\
$0.0$ & $-1816.2$ & $-210.1$ & $-712.5$ & $-859.5$ & $-1515.9$ & $-692.6$ & $-3555.2$ & $-3244.0$ & $-13367.3$ & $-13260.3$ & $-746.2$ & $-360.8$ & $-392.8$ & $-380.8$ & $-339.0$\\
$-10.1$ & $-210.1$ & $-855.4$ & $-235.2$ & $-238.4$ & $-409.0$ & $-192.0$ & $-989.2$ & $-877.3$ & $-3472.3$ & $-3448.4$ & $38.8$ & $-77.7$ & $-130.5$ & $-114.6$ & $-97.2$\\
$-404.2$ & $-712.5$ & $-235.2$ & $-1550.4$ & $-824.9$ & $-1158.3$ & $-596.2$ & $-3035.1$ & $-3077.9$ & $-11335.2$ & $-11372.4$ & $-590.5$ & $-256.3$ & $-345.1$ & $-351.5$ & $-228.0$\\
$-421.3$ & $-859.5$ & $-238.4$ & $-824.9$ & $-966.7$ & $-1478.0$ & $-700.8$ & $-3622.3$ & $-3394.8$ & $-13681.8$ & $-13590.3$ & $-788.6$ & $-375.8$ & $-390.7$ & $-386.2$ & $-337.6$\\
$-686.3$ & $-1515.9$ & $-409.0$ & $-1158.3$ & $-1478.0$ & $-2676.7$ & $-1191.0$ & $-6153.7$ & $-5527.5$ & $-23043.4$ & $-22841.3$ & $-1476.4$ & $-657.8$ & $-661.1$ & $-642.5$ & $-570.3$\\
$-332.0$ & $-692.6$ & $-192.0$ & $-596.2$ & $-700.8$ & $-1191.0$ & $-632.8$ & $-2837.2$ & $-2588.6$ & $-10605.8$ & $-10519.4$ & $-667.3$ & $-298.9$ & $-307.6$ & $-300.6$ & $-262.4$\\
$-1720.7$ & $-3555.2$ & $-989.2$ & $-3035.1$ & $-3622.3$ & $-6153.7$ & $-2837.2$ & $-14925.2$ & $-13444.8$ & $-55598.2$ & $-55158.9$ & $-3253.8$ & $-1563.4$ & $-1594.9$ & $-1564.1$ & $-1353.3$\\
$-1655.1$ & $-3244.0$ & $-877.3$ & $-3077.9$ & $-3394.8$ & $-5527.5$ & $-2588.6$ & $-13444.8$ & $-12906.4$ & $-50808.4$ & $-50436.9$ & $-2877.7$ & $-1404.3$ & $-1458.9$ & $-1435.2$ & $-1228.6$\\
$-6608.0$ & $-13367.3$ & $-3472.3$ & $-11335.2$ & $-13681.8$ & $-23043.4$ & $-10605.8$ & $-55598.2$ & $-50808.4$ & $-211160.6$ & $-209177.6$ & $-12006.6$ & $-5906.8$ & $-5995.0$ & $-5889.2$ & $-5087.2$\\
$-6563.0$ & $-13260.3$ & $-3448.4$ & $-11372.4$ & $-13590.3$ & $-22841.3$ & $-10519.4$ & $-55158.9$ & $-50436.9$ & $-209177.6$ & $-208012.2$ & $-11907.2$ & $-5837.5$ & $-5948.3$ & $-5859.1$ & $-5058.2$\\
$-390.1$ & $-746.2$ & $38.8$ & $-590.5$ & $-788.6$ & $-1476.4$ & $-667.3$ & $-3253.8$ & $-2877.7$ & $-12006.6$ & $-11907.2$ & $-1680.5$ & $-352.0$ & $-367.2$ & $-362.3$ & $-319.3$\\
$-163.8$ & $-360.8$ & $-77.7$ & $-256.3$ & $-375.8$ & $-657.8$ & $-298.9$ & $-1563.4$ & $-1404.3$ & $-5906.8$ & $-5837.5$ & $-352.0$ & $-800.3$ & $-0.0$ & $0.0$ & $-0.0$\\
$-189.3$ & $-392.8$ & $-130.5$ & $-345.1$ & $-390.7$ & $-661.1$ & $-307.6$ & $-1594.9$ & $-1458.9$ & $-5995.0$ & $-5948.3$ & $-367.2$ & $-0.0$ & $-815.6$ & $0.0$ & $-0.0$\\
$-211.5$ & $-380.8$ & $-114.6$ & $-351.5$ & $-386.2$ & $-642.5$ & $-300.6$ & $-1564.1$ & $-1435.2$ & $-5889.2$ & $-5859.1$ & $-362.3$ & $0.0$ & $0.0$ & $-800.0$ & $0.0$\\
$-170.0$ & $-339.0$ & $-97.2$ & $-228.0$ & $-337.6$ & $-570.3$ & $-262.4$ & $-1353.3$ & $-1228.6$ & $-5087.2$ & $-5058.2$ & $-319.3$ & $-0.0$ & $-0.0$ & $0.0$ & $-695.0$\\
\par \toprule
\end{tabular}

    \end{table}
\end{landscape}



\end{document}

